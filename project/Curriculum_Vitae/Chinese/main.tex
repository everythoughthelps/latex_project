%%%%%%%%%%%%%%%%%%%%%%%%%%%%%%%%%%%%%%%%%
% "ModernCV" CV and Cover Letter
% LaTeX Template
% Version 1.1 (9/12/12)
%
% This template has been downloaded from:
% http://www.LaTeXTemplates.com
%
% Original author:
% Xavier Danaux (xdanaux@gmail.com)
%
% License:
% CC BY-NC-SA 3.0 (http://creativecommons.org/licenses/by-nc-sa/3.0/)
%
% Important note:
% This template requires the moderncv.cls and .sty files to be in the same 
% directory as this .tex file. These files provide the resume style and themes 
% used for structuring the document.
%
%%%%%%%%%%%%%%%%%%%%%%%%%%%%%%%%%%%%%%%%%

%----------------------------------------------------------------------------------------
%	PACKAGES AND OTHER DOCUMENT CONFIGURATIONS
%----------------------------------------------------------------------------------------

\documentclass[11pt,a4paper,sans]{moderncv} % Font sizes: 10, 11, or 12; paper sizes: a4paper, letterpaper, a5paper, legalpaper, executivepaper or landscape; font families: sans or roman
\usepackage{xeCJK}
\setCJKmainfont{Droid Sans Fallback}
\setCJKsansfont{WenQuanYi Zen Hei}
\setCJKmonofont{cwTeXFangSong}
\moderncvstyle{classic} % CV theme - options include: 'casual' (default), 'classic', 'oldstyle' and 'banking'
\moderncvcolor{blue} % CV color - options include: 'blue' (default), 'orange', 'green', 'red', 'purple', 'grey' and 'black'

\usepackage{lipsum} % Used for inserting dummy 'Lorem ipsum' text into the template

\usepackage[scale=0.85]{geometry} % Reduce document margins
%\setlength{\hintscolumnwidth}{3cm} % Uncomment to change the width of the dates column
%\setlength{\makecvtitlenamewidth}{10cm} % For the 'classic' style, uncomment to adjust the width of the space allocated to your name

%----------------------------------------------------------------------------------------
%	NAME AND CONTACT INFORMATION SECTION
%----------------------------------------------------------------------------------------

\firstname{潘孟} % Your first name
\familyname{} % Your last name

% All information in this block is optional, comment out any lines you don't need
%\title{Curriculum Vitae}
\address{邮编510006}{广州市番禺区大学城外环东路132号}
\mobile{(+86) 18566086908}
%\phone{(000) 111 1112}
%\fax{(000) 111 1113}
\email{panm9@mail2.sysu.edu.cn}
%\homepage{staff.org.edu/~jsmith}{staff.org.edu/$\sim$jsmith} % The first argument is %the url for the clickable link, the second argument is the url displayed in the %template - this allows special characters to be displayed such as the tilde in this %example
%\extrainfo{additional information}
%\photo[70pt][0.4pt]{picture} % The first bracket is the picture height, the second is %the thickness of the frame around the picture (0pt for no frame)
\quote{}

%----------------------------------------------------------------------------------------

\begin{document}
\makecvtitle % Print the CV title

%----------------------------------------------------------------------------------------
%	EDUCATION SECTION
%----------------------------------------------------------------------------------------

\section{学历}

\cventry{2013--2017}{交通运输}{交通运输学院}{石家庄铁道大学}{}{
}  % Arguments not required can be left empty
\cventry{2018--至今}{检测技术与自动化装置}{智能工程学院}{中山大学}{}{
}
\vspace{-0.8cm}
%----------------------------------------------------------------------------------------
%	WORK EXPERIENCE SECTION
%----------------------------------------------------------------------------------------
\section{研究经历}
%
\cventry{2019.7--2019.11}{对常规显著信号提取的pooling改进:different pooling}{}{}{}{
\begin{itemize}
\item \href{address}{https://github.com/everythoughthelps/different\_pooling}
\item 将传统的提取核内最显著信号的maxpooling改为提取特征值和核内平均特征值相差最大的different pooling。
\end{itemize}
}
\vspace{0.5cm}
\cventry{2019.11--2020.5}{基于像素分类的单目深度估计网络}{CAC2020独立一作}{}{}{
\begin{itemize}
\item \href{address}{https://github.com/everythoughthelps/depth-estimation}
\item 将深度图预测视为像素在通道维度上的分类问题。网络骨干使用unet,在decoder上添加
特征注意力模块(Feature Attention Module),在最后的输出层将像素值与通道信息解耦合,在NYUv2深度数据集获得了RMSE 0.579的结果。
\end{itemize}
}
\vspace{0.5cm}
\cventry{2020.7-2021.1}{Self-distillation Network for Indoor and Outdoor 
Monocular Depth Estimation}{在审}{Multimedia Tools and Applications}{}{
\begin{itemize}
\item \href{address}{https://github.com/everythoughthelps/SDMDE}
\item 为了解决单目深度估计的鲁棒性问题,我们首创性地使用室内、室外数据集对网络进行同时训
练。
\item 实验表明单纯的使用复合数据集并没有使网络的性能提升,相反会出现退化现象,这是由于多样的
数据分布对网络的拟合能力和表达能力是一种挑战,所以我们提出了自蒸馏单目深度估计框架。
实验证明我们的框架对网络的表达能力和拟合能力均有一定的提升,减弱了网络在复合数据集上的退化现
象。
\end{itemize}
}

%----------------------------------------------------------------------------------------
%	COMPUTER SKILLS SECTION
%----------------------------------------------------------------------------------------

\section{专业能力}
\cvitem{}{熟悉Python,对C++,shell有一定了解}
\cvitem{}{有丰富Linux系统开发经验,熟练使用pycharm,vim等开发工具}
\cvitem{}{熟练使用Git以及Github等代码管理工具进行团队协作开发}
\cvitem{}{熟练使用Pytorch深度学习框架,对TensorFlow有一定了解}
\cvitem{}{熟练使用OpenCV,PIL,Numpy,Matplotlib等模块}
\cvitem{}{熟练使用LaTeX,丰富的论文编辑经验}

%----------------------------------------------------------------------------------------
%	LANGUAGES SECTION
%----------------------------------------------------------------------------------------

\section{语言}
%\cvitem{英文} {四级、 六级、 雅思6.0}
\cventry{}{四级}{六级}{雅思6.0}{}{}
%----------------------------------------------------------------------------------------
%	INTERESTS SECTION
%----------------------------------------------------------------------------------------
%\bigskip

\section{其他经历}
\cvitem{}{虚拟现实技术助教,主要负责Unity软件的教学。}
\cvitem{}{深度学习服务器管理员,主要负责学院所有深度学习服务器的维护,包括网络维护,磁盘管理,环境搭建等。}{}{}

%----------------------------------------------------------------------------------------
%	COVER LETTER
%----------------------------------------------------------------------------------------

% To remove the cover letter, comment out this entire block

%\clearpage

%\recipient{HR Departmnet}{Corporation\\123 Pleasant Lane\\12345 City, State} % Letter recipient
%\date{\today} % Letter date
%\opening{Dear Sir or Madam,} % Opening greeting
%\closing{Sincerely yours,} % Closing phrase
%\enclosure[Attached]{curriculum vit\ae{}} % List of enclosed documents

%\makelettertitle % Print letter title

%\lipsum[1-3] % Dummy text

%\makeletterclosing % Print letter signature

%----------------------------------------------------------------------------------------
\end{document}