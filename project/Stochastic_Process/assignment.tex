\documentclass{assignment}


%\coursetitle{Creating assignments}
%more class  https://ctan.org/topic/class
%more template https://www.latexstudio.net/
\courselabel{English Academic Writing}
\exercisesheet{Final Project}{Documentation}
\student{Pan Meng}
\semester{Fall 2021}
\date{Feb 11, 2022}
\finishdate{Feb 10, 2022}
%\usepackage[pdftex]{graphicx}
%\usepackage{subfigure}
\begin{document}
% 根据输入的第一个参数围来读取图片显示的写法
\logo{logo.jpg}
% 固定死是logo.jpg的写法
% \logo

\begin{problemlist}
\pbitem 


\begin{problem}
\begin{answer}


  As one of the most crucial tasks of scene perception, monocular depth estimation (MDE) has made
  considerable development in recent years. Current MDE researchers are interested in the precision and speed of
  the estimation, but ignore the generalization ability across scenes for quite a long time. For instance, a MDE
  network trained on outdoor scenes achieve impressive performance on outdoor scenes but poor performance on
  indoor scenes, vice versa. In this paper, we propose a self-distillation MDE network to improve the generalization
  ability across different scenes. Specifically, we design a student encoder that extracts features from two datasets of
  indoor and outdoor scenes, respectively. After that, we introduce a dissimilarity loss to pull apart encoded features
  of different scenes in the feature space. Finally, a decoder is adopted to estimate the final depth from encoded
  features. In doing so, our self-distillation MDE network can learn the depth estimation of two different datasets. To
  our best knowledge, we are the first one to tackle the generalization problem across datasets of different scenes in
  the MDE field. Experiments demonstrate that our method achieves competitive estimation performance, compared
  with state-of-the-art MDE methods. Note that evaluating on two datasets by a single network is more challenging
  than evaluating on two datasets by two different networks.
\end{answer}
\end{problem}

%\pbitem What does the preamble contain ?



%The preamble may contain the following declarations\footnote{Current markup's preamble.} :

%\begin{verbatim}
%\documentclass{assignment}
%\coursetitle{Creating assignments}
%\courselabel{ASG 101}
%\exercisesheet{Home Work 1}{Documentation}
%\student{Madhusudan Singh}
%\semester{Summer 2004}
%\date{July 14, 2004}
%%\usepackage[pdftex]{graphicx}
%%\usepackage{subfigure}
%\end{verbatim}


%Its possible. Just pass the options in the preamble :

%\begin{verbatim}
%\documentclass[option1,option2, ...]{assignment}
%\end{verbatim}


%Equation numbers refer to the problem number. For instance,

%\begin{verbatim}
%\begin{eqnarray}
%E & = & mc^{2} \label{eqn:emc2} \\
%\textrm{That is how equations are numbered} \ldots \label{eqn:numbered} \\
%\textrm{Or not numbered} \ldots \nonumber
%\end{eqnarray}
%\end{verbatim}
%
%\begin{eqnarray}
%E & = & mc^{2} \label{eqn:emc2} \\
%\textrm{That is how equations are numbered} \ldots \label{eqn:numbered} \\
%\textrm{Or not numbered} \ldots \nonumber
%\end{eqnarray}




%\begin{verbatim}
%\begin{answer}
%\begin{eqnarray}
%Answer=f(bold) \nonumber
%\end{eqnarray}
%\end{answer}
%\end{problem}
%\end{verbatim}


%\pbitem Can one have more than one answer section for the problem ?
%
%\begin{problem}
%
%Most definitely.
%
%Certain problems have many parts :
%
%\begin{enumerate}
%\item Part 1
%
%\begin{answer}
%Answer to part one.
%\end{answer}
%
%\item Part 2
%
%\begin{answer}
%Answer to part two.
%\end{answer}
%
%\end{enumerate}
%
%\begin{verbatim}
%\begin{enumerate}
%\item Part 1
%
%\begin{answer}
%Answer to part one.
%\end{answer}
%
%\item Part 2
%
%\begin{answer}
%Answer to part two.
%\end{answer}
%
%\end{enumerate}
%
%\end{verbatim}
%
%\end{problem}
%
%
%\pbitem What are the copyright conditions for this class file ?
%
%\begin{problem}
%
%This material is subject to the \LaTeX\ Project Public License. See http://www.ctan.org/tex-archive/help/Catalogue/licenses.lppl.html for the details of that license. See the LICENSE file for more details.
%
%
%\begin{answer}
%\begin{eqnarray}
%&& Q. E. D. \nonumber
%\end{eqnarray}
%\end{answer}
%
%\end{problem}
%
%
%\pbitem How do I get help using this class ?
%\begin{problem}
%
%
%You may post your queries on comp.text.tex . I check it fairly regularly.
%\end{problem}

\end{problemlist}

\end{document}
