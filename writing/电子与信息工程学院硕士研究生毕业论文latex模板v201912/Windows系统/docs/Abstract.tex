%%
% 摘要信息
% 摘要内容应概括地反映出本论文的主要内容,主要说明本论文的研究目的、内容、方法、成果和结论。要突出本论文的创造性成果或新见解,不要与引言相 混淆。语言力求精练、准确,以 300—500 字为宜。
% 关键词是供检索用的主题词条,应采用能覆盖论文主要内容的通用技术词条(参照相应的技术术语 标准)。按词条的外延层次排列(外延大的排在前面)。


\cabstract{
	随着人工智能技术的迅猛发展,人们对智能感知算法提出了更高的要求,
	如何快速获得精确的场景信息成为学者们研究的重点。
	其中场景的深度信息作为众多任务的关键信息,在诸多场景
	如自动驾驶,3D物体检测,三维重建中
	显得尤为重要。单目深度估计,即从单张RGB图像
	中获取场景的深度具有很多优点:对硬件要求低,能耗低,
	便于搭载等。但是单目深度估计对算法本身要求极高,因为
	RGB图像在成像过程中将3D信息完全丢失,没有其他信息辅助的情况下,
	很难将丢失的3D信息恢复。

	本文首先对基于深度学习的单目深度估计技术进行了
	调研,阐述了国内外的研究现状。随后对本文依赖的关键技术
	进行简要讲解,包括深度学习算法,单目深度估计技术,
	双目深度估计技术等。随后重点描述
	围绕单目深度估计任务提出了两种基于深度学习算法的解决方案。
	第一种将单目深度估计建模为像素分类问题,提出了
	一种端到端的编解码网络,通过预测像素
	在每个深度区间的置信度来进行深度估计。在解码过程中加入了
	特征注意力模块,该模块可以使上采样网络对不同来源
	的特征图进行加权,随后应用在重建过程中。客观实验
	表明该算法在室内室外数据集均获得了较好的结果。

	本文重点关注到单目深度估计网络在复杂多样场景上的鲁棒问题,
	提出使用室内外复合数据集对网络进行训练。但是复合数据集
	具有完全不同的数据分布,一方面网络很难拟合这些分布,
	另一方面网络将多样的重建效果表达出来也同样困难。
	为了解决这个问题本文在第四章重点描述了第二种解决方案:
	自蒸馏单目深度估计框架。这种框架对室内外数据集同时采样,
	两个场景的样本被送入不同的编码器,产生各个场景的特征图。
	并提出了相异性损失
	函数来扩大两个场景的特征图的间距,达到扩大
	数据集类别外间距的效果,使网络可以针对不同的场景提取
	特定的特征。对比实验表明该框架有效地解决了网络在面对
	多样场景时的性能退化现象。适用多样数据集时,
	自蒸馏深度估计框架使所有指标均获得一定的提升,其中
	某些指标甚至超越了
	在单一数据集
	上的表现,达到了最优效果。更重要的是该框架适用于
	大部分编解码网络,移植性强。
}
% 中文关键词(每个关键词之间用“;”分开,最后一个关键词不打标点符号。)
\ckeywords{单目深度估计;像素分类;特征注意力模块;自蒸馏;多样场景 }

\eabstract{
	With the rapid development of artificial intelligence 
technology,  people put forward higher requirements 
for intelligent sensing algorithms. Obtaining 
accurate scene information quickly has become 
the focus of research. Depth information of the 
scene is wildly used in many tasks, 
such as autopilot, 3D object detection, 3D reconstruction.
Monocular depth estimation, obtaining the depth 
of the scene from a single RGB image, has many advantages, 
such as lower hardware requirements, less energy consumption, 
and easy to carry. However, monocular depth estimation has a
high demand on the algorithm itself, because the 3D information
 of RGB image is completely lost in the imaging process,
  and it is difficult to recover the 3D information without
   other information assistance. 

 Firstly, this paper investigates (MDE)
  technology based on deep learning techniques, 
  and describes the 
  domestic and foreign research status. Then the key 
  technologies of this paper are briefly explained, 
  including deep learning algorithms, monocular depth 
  estimation technologes, binocular depth estimation 
  technologies and so on. Then, focusing on the monocular depth 
  estimation task, two deep learning-based solutions 
  algorithm are proposed.
  In the first method, monocular depth estimation is 
  modeled as a pixel classification problem, 
  and an end-to-end encode-decode network is proposed 
  to estimate depth by predicting the confidence of 
  pixels in each depth interval. In the decoding process, 
  the feature attention module is loaded, 
  which enables the up sampling network to weight 
  the feature maps from different sources, and then 
  applies them in the reconstruction process. 
  Quantative experiments show that the algorithm 
  achieves good results on both indoor and outdoor dataset.

  We focus on the robust performance of MDE networks,
  to tackle this problem, we propose a self-distillation MDE network to 
  improve the generalization ability across different scenes. 
  Specifically, we design a student encoder that extracts features from 
  two datasets of indoor and outdoor scenes, respectively. After that, 
  we introduce a dissimilarity loss to pull apart encoded features of different 
  scenes in the feature space. In doing so, our self-distillation 
  MDE network can learn the depth estimation of two different datasets. 
  To our best knowledge, we are the first one to tackle the generalization problem across datasets of different scenes. Experiments demonstrate that our method achieves competitive estimation performance. 
}
% 英文文关键词(关键词之间用逗号隔开,最后一个关键词不打标点符号。)
\ekeywords{monocular depth estimation, pixel classification, 
feature attention module, self-distillation, various scenes}