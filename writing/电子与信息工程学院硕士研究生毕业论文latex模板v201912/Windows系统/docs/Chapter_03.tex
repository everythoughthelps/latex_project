\chapter{基于像素分类的单目深度估计网络}
\section{引言}
深度图为每个像素存储了场景到相机距离信息的图像。深度估计任务
也就是估计给每个像素上场景到相机的距离,这与像素分类任务十分相似。本文将
深度估计任务建模为像素级别的分类任务,提出了一种单目深度估计
算法。首先需要将真实稠密的深度图进行离散化,使其成为离散的深度标签,
随后算法使用一张RGB图像作为输入,经过重复的特征图提取卷积层后输出了
不同尺度的特征图,这些特征图通过U-Net\cite{unet}
短路连接(ShortCut)参与到上采样重建过程。
近年来许多方法使用短路连接将编码端的特征图直接输出到解码端,
通过短路连接输入的特征图和上采样特征图对重建过程的贡献并不
是相同的,为了使网络对短路连接输入的特征图和上采样特征图
有可学习的关注度,本章设计了特征注意力模块(Feature
Attention Module, FAM)。最后网络输出了每个像素位置
的备选深度及其相应概率,并通过计算深度与对应概率的期望获得最终深度。
网络架构如图\ref{Stream of class}所示。

本章的创新点总结如下:
\begin{enumerate}
  \item 将深度估计任务建模为像素分类任务,提出了一种像素分类单目深度估计框架。
  \item 提出了特征注意力模块,使网络可以学习对特征图的注意程度。
  \item 实验结果表明本章网络性能可以达到国内外先进水平,且特征注意力
  模块对重建性能有一定的提升效果。
\end{enumerate}
\begin{figure}[htb]
    \centering
    \includegraphics[width=0.9\linewidth]{figure/Stream_of_class.pdf}
    \caption{基于像素分类的单目深度估计网络架构\cite{pm}}
    \label{Stream of class}
\end{figure}
\section{基于像素分类的单目深度估计网络}
\subsection{网络结构}
\begin{figure}[htb]
    \centering
    \includegraphics[width=0.9\linewidth]{figure/Architecture.pdf}
    \caption{像素分类单目深度估计网络的具体网络结构}
    \label{cls_architecture}
\end{figure}
本章提出的具体网络结构如图\ref{cls_architecture}所示。
网络采用了编解码结构,骨干网络上使用了类U-Net结构。
在编码端,RGB图像首先经过一个$3\times3$步长为1
的卷积核,使
图像的通道扩展到64,随后再次经过一个$3\times3$步长为1
的卷积核,通道维持不变。下采样方式设置了步长为2的
最大池化层。该卷积、下采样组合网络重复堆叠五次,每次获得的特征图通道
分别为64,128,256,512,1024。即第一次将RGB图像拓展到
64通道,随后四次下采样均伴随着尺寸减半,通道加倍。
最小尺度的特征图尺寸为$1024\times19\times15$,
接着经过步长为2的最大池化层,使用卷积层对其通道加倍后
进入解码端对特征图进行上采样重建。
网络使用短路连接将编码的多尺度
特征图直接输入到解码端进行特征融合。高维度特征图和编码特征图共同参与到
上采样过程中,为了使网络可以学习对每个特征图的关注程度,
本章提出了特征注意力模块。经过特征注意力模块后,使用步长为1
,$3\times3$的卷积核进行卷积计算。经过四次特征注意力模块后
得到原尺寸$\frac{1}{2}$的特征图,随后使用一次上采样
和一个步长为1,$3\times3$的卷积核将特征图重建为原尺寸大小,
通道数为像素分类的类别C。对最终输出在通道维度内使用softmax
计算,得到每个备选深度的置信度。最后依靠置信度计算最终深度的期望。

\subsection{特征注意力模块}
本章使用了类U-Net的骨干网络来实现像素分类单目深度估计网络,
网络提取的多尺度特征
通过短路连接直接加入到重建过程中。重建时,网络依靠多尺度
特征和解码特征共同完成重建预测。而两个特征
对重建的贡献是不同的,所以本章设计了特征注意力模块,使
网络可以学习对每个特征图的关注程度。
特征注意力模块的结构如图\ref{FAM}所示。
模块首先将多尺度特征和高维度的特征在通道维度
上连接起来,随后经过全局平均池化层,获得一个一维向量。
该向量代表了通道维度上的显著信号,随后向量经过
$1\times1$的卷积层,ReLu激活层,sigmoid激活层
后与连接后的特征图内积计算。
连接后的特征图同时被直接输出,与内积结果求和,作为最终的输出。
\begin{figure}[htb]
\centering
\includegraphics[width=0.4\linewidth]{figure/FAM.pdf}
\caption{特征注意力模块网络结构}
\label{FAM}
\end{figure}
\subsection{损失函数}
基于像素分类的单目深度估计网络在最后一层需要进行像素分类。
从C个深度类别中挑选最接近真实深度的像素。
网络使用交叉熵损失函数(Cross-Entropy Loss)
来监督网络的分类过程。
\begin{align}
    \mathcal{L} = \frac{1}{N}\sum_N \sum_{i=0}^{C} -i \log (p_i)
\end{align}
式中N为图像中有效的像素数目,$i$与别$p_i$分别为标签和该标签的概率。
$C$为备选的深度类别。
\section{实验结果与分析}
为了更全面的评价本章所提出的网络和特征注意力模块带来的提升,
本节设计了对比实验,从定性对比和视觉对比两个方面
对网络的表现进行了对比。
\subsection{实现细节}
本节针对实验的细节进行详细阐述。本章所有程序均
使用PyTorch\cite{paszke2019pytorch}
开源深度学习框架进行实现。
网络采用了4张RTX2080TI进行训练,NYU depth v2和KITTI的
批数量分别设置为24和16,初始学习率设置为0.001,
并且每10次迭代折半衰减一次,总共设置了100次迭代。
选取了Adam\cite{2014adam}作为优化器,
momentum和weight decay分别设置为0.9和0.0005。
\subsection{数据集与评价指标}
本节在实验时组织了多个数据集,包括
室内数据集NYU depth和室外数据集KITTI。这两个数据集分别是
单目深度估计在室内场景和室外场景最常用的数据集。
NYU depth数据集是常用的深度估计数据集,它主要关注室内环境,
数据集由464个室内场景组成,
其中249个场景用作训练,215个场景用作测试。不同于KITTI数据集
使用激光雷达来采集深度信息,NYU depth数据集使用
Microsoft的Kinect RGB-D相机来采集视频序列和对应的深度。
图像和深度图的分辨率为$480 \times 640$,通常会在实验中
下采样减半来加速训练过程。为了去掉冗余的边缘信息,
还会将图像进行中心裁剪到$228 \times 304$。
因为不同的RGB相机和深度相机之间
不是完全同步的,深度图和RGB图像之间不是一对一的对应关系。
为了对齐深度图和RGB图像,每个深度图都与最近的RGB图像同步并且
关联在一起。由于硬件原因,采集到的深度图通常会有空洞,
这些空洞可以
使用数据集提供的工具箱进行填补。

KITTI数据集由装备在一台汽车上的传感器来进行采集,采集到的
信息包括双目彩色视频,双目黑白视频,雷达点云信息,惯性导航信息,
GPS信息等。原始数据量十分庞大。每个场景都包含一对双目RGB
图像和对应的深度信息,分辨率为$1224 \times 368$。在使用上,KITTI有两种不同的划分。
其中在Eigen Split选取了61个场景中的28个作为测试场景,
从中选出了697张图像用来测试。在另外的28个场景中选出
了26000张图像用于训练。而 KITTI split 包含29,000张训练图像,
200张测试图像。

对比实验选取了单目深度估计中的关键技术方法进行对比,并且
选取了最常用的五个评价指标\cite{eigen2014depth}衡量本章提出的方法:
$
 RMSE=\sqrt{\frac{1}{\lvert N \rvert} \sum\nolimits_{i\in N}\lvert|d_i - d_i^*\rvert|^2} 
$,
$
  RMSE \ log = \sqrt{\frac{1}{\lvert N \rvert}\sum\nolimits_{i\in N}\lvert| \lg(d_i) - \lg(d_i^*) \rvert|^2}
$,
$
  Abs \ Rel=\frac{1}{\lvert N \rvert}\sum\nolimits_{i \in N}\frac{\lvert d_i - d_i^* \rvert}{d_i^*}
$,
$
  Sq \ Rel=\frac{1}{\lvert N \rvert}\sum\nolimits_{i \in N}\frac{\lvert| d_i - d_i^* \rvert|^2}{d_i^*}
$,
$
  Accuracies:\% \ of \ d_i \ s.t. \max(\frac{d_i}{d_i^*},\frac{d_i^*}{d_i})=\delta < thr 
$。
其中$d_i$和$d_i^*$表示预测深度与真实深度。$N$代表
所有的有效像素。
\subsection{定量结果}
\begin{table*}[htb]
    \centering
    \caption{单目深度估计网络在NYU depth数据集上的效果对比,最佳结果使用加粗处理}
    \label{tab:nyu cls quantitative result}
    \begin{tabular}{c|cccc|ccc}
      \toprule
      \multirow{2}{*}{方法} & \multicolumn{4}{c}{越低效果越好}&\multicolumn{3}{|c}{越高效果越好}\\
      & Rel Abs & Rel Sq & RMSE& RMSE $log$ &$\delta<1.25$ &$\delta<1.25^2$ & $\delta<1.25^3$ \\   
      \midrule            
      Saxena\cite{Make3D}&0.349&-&1.214&0.430&0.447&0.745&0.897\\
      Eigen\cite{eigen2014depth}&0.158&0.218&0.641&0.285&0.769&0.950&0.962\\
      Li\cite{li2015depth}&0.152&0.232&0.611&0.759&0.789&0.955&0.987\\
      chakrabarti\cite{chakrabarti2016depth}&0.149&0.118&0.620&-&0.806&0.953&0.988\\
      Laina\cite{laina2016deeper}&0.194&0.101&0.584&-&0.811&0.953&0.988\\
      Xu\cite{xu2018structured}&\textbf{0.125}&0.135&0.597&0.593&0.806&0.952&0.986\\
      Lee\cite{lee2019monocular}&0.131&\textbf{0.087}&\textbf{0.538}&\textbf{0.180}&\textbf{0.837}&\textbf{0.971}&\textbf{0.994}\\
      Ours(VGG)&0.141&0.115&0.610&0.445&0.798&0.942&0.988\\
      Ours(Res101)&0.138&0.113&0.597&0.367&0.812&0.963&\textbf{0.994}\\
      \bottomrule
    \end{tabular}
  \end{table*}
  \begin{table*}[htb]
    \centering
    \caption{单目深度估计网络在KITTI数据集上的效果对比,最佳结果使用加粗处理}
    \label{tab:kitti cls quantitative result}
    \begin{tabular}{c|cccc|ccc}
      \toprule
      \multirow{2}{*}{方法} & \multicolumn{4}{c}{数值越低效果越好}&\multicolumn{3}{|c}{数值越高效果越好}\\
      & Rel Abs & Rel Sq & RMSE& RMSE $log$ &$\delta<1.25$ &$\delta<1.25^2$ & $\delta<1.25^3$ \\   
      \midrule            
      Saxena\cite{Make3D}&0.280&3.012&8.734&0.361&0.601&0.820&0.928\\
      Eigen\cite{eigen2014depth}&0.190&1.515&7.156&0.270&0.692&0.899&0.967\\
      Liu\cite{liu2015learning}&0.217&1.841&6.986&0.289&0.647&0.882&0.961\\
      Kuznietsov\cite{kuznietsov}&0.113&0.741&4.621&0.189&0.862&0.960&0.986\\
      Chen\cite{2019semantic}&0.102&0.890&5.203&0.183&0.863&0.955&0.984\\
      Xu\cite{xu2018structured}&0.122&0.897&4.677&-&0.818&0.954&0.985\\
      Fu\cite{FuCVPR18-DORN}&\textbf{0.072}&0.376&\textbf{3.056}&\textbf{0.132}&\textbf{0.915}&\textbf{0.980}&\textbf{0.993}\\
      Ours(VGG)&0.120&0.144&4.737&0.207&0.837&0.947&0.964\\
      Ours(Res101)&0.117&\textbf{0.142}&4.643&0.198&0.809&0.952&0.964\\
      \bottomrule
    \end{tabular}
  \end{table*}
本节针对网络在KITTI和NYU depth两个数据集上的表现进行了对比实验,
实验选取了领先性能的方法和具有里程碑性质的方法进行了对比。
NYU depth数据集的表现对比如表
\ref{tab:nyu cls quantitative result}所示。
本章所提出方法在各项指标上都达到了领先水平,其中在$\delta<1.25^3$
的指标上达到了最优。相比骨干网络使用VGG,骨干网络使用ResNet101
取得的效果普遍更加优秀,大约有2.1\%的效果提升。

本节还在KITTI数据集上对网络进行了测试,各项指标表现如表
\ref{Pixel_cla_KITTI_visualization_result}所示。
实验选取了Saxena\cite{Make3D}、Eigen\cite{eigen2014depth}、
Liu\cite{liu2015learning}、Kuznietsov\cite{kuznietsov}
与近两年的领先方法Xu\cite{xu2018structured}、
Fu\cite{FuCVPR18-DORN}作为基准。与
室内数据集NYU depth相比,像素分类网络在Sq rel表现更好,
达到了方法最优结果。Xu等人\cite{xu2018structured}
在Rel Abs, RMSE RMSE$\log$,以及另外误差像素所占百分比几个
指标上表现优异,达到了最优结果。最后两行分别展示了使用VGG
和ResNet101作为骨干网络的结果,ResNet网络表现仍然优于
VGG网络,大约有2\%的效果提升。在错误像素所占百分比指标上,
三种指标表现均低于室内数据集NYU depth,这可能与室外数据集
深度范围较大,类别间的深度相差较大造成的。

\subsection{视觉结果}
\begin{figure*}[htb]
  \centering
  \begin{subfigure}{0.24\linewidth}
  \begin{minipage}[b]{1\linewidth}
  \includegraphics[width=1\linewidth]{figure/Pixel_cla_kitti/1rgb.png}\vspace{4pt}
  \includegraphics[width=1\linewidth]{figure/Pixel_cla_kitti/2rgb.png}\vspace{4pt}
  \includegraphics[width=1\linewidth]{figure/Pixel_cla_kitti/3rgb.png}\vspace{4pt}
  \includegraphics[width=1\linewidth]{figure/Pixel_cla_kitti/4rgb.png}
  \end{minipage}
  \caption{RGB图像}
  \end{subfigure}
  \begin{subfigure}{0.24\linewidth}
  \begin{minipage}[b]{1\linewidth}
  \includegraphics[width=1\linewidth]{figure/Pixel_cla_kitti/1gt.png}\vspace{4pt}
  \includegraphics[width=1\linewidth]{figure/Pixel_cla_kitti/2gt.png}\vspace{4pt}
  \includegraphics[width=1\linewidth]{figure/Pixel_cla_kitti/3gt.png}\vspace{4pt}
  \includegraphics[width=1\linewidth]{figure/Pixel_cla_kitti/4gt.png}
  \end{minipage}
  \caption{真实深度图}
  \end{subfigure}
  \begin{subfigure}{0.24\linewidth}
  \begin{minipage}[b]{1\linewidth}
  \includegraphics[width=1\linewidth]{figure/Pixel_cla_kitti/1res.png}\vspace{4pt}
  \includegraphics[width=1\linewidth]{figure/Pixel_cla_kitti/2res.png}\vspace{4pt}
  \includegraphics[width=1\linewidth]{figure/Pixel_cla_kitti/3res.png}\vspace{4pt}
  \includegraphics[width=1\linewidth]{figure/Pixel_cla_kitti/4res.png}
  \end{minipage}
  \caption{本章方法}
  \end{subfigure}
  \begin{subfigure}{0.24\linewidth}
  \begin{minipage}[b]{1\linewidth}
  \includegraphics[width=1\linewidth]{figure/Pixel_cla_kitti/1eigen.png}\vspace{3.5pt}
  \includegraphics[width=1\linewidth]{figure/Pixel_cla_kitti/2eigen.png}\vspace{3.5pt}
  \includegraphics[width=1\linewidth]{figure/Pixel_cla_kitti/3eigen.png}\vspace{3.5pt}
  \includegraphics[width=1\linewidth]{figure/Pixel_cla_kitti/4eigen.png}
  \end{minipage}
  \caption{Liu\cite{liu2015learning}}
  \end{subfigure}
  \caption{像素分类单目深度估计网络在KITTI数据集上的可视化结果,
  (a)为RGB图像,(b)为真实的深度标注,(c)为本章方法
  预测结果,(d)为Liu\cite{liu2015learning}预测结果。}
  \label{Pixel_cla_KITTI_visualization_result}
  \end{figure*}
\begin{figure*}[htb]
  \centering
  \begin{subfigure}{0.24\linewidth}
  \begin{minipage}[b]{1\linewidth}
  \includegraphics[width=1\linewidth]{figure/Pixel_cla_nyu/1rgb.jpg}\vspace{4pt}
  \includegraphics[width=1\linewidth]{figure/Pixel_cla_nyu/2rgb.jpg}\vspace{4pt}
  \includegraphics[width=1\linewidth]{figure/Pixel_cla_nyu/3rgb.jpg}\vspace{4pt}
  \includegraphics[width=1\linewidth]{figure/Pixel_cla_nyu/4rgb.jpg}\vspace{4pt}
  \includegraphics[width=1\linewidth]{figure/Pixel_cla_nyu/5rgb.jpg}
  \end{minipage}
  \caption{RGB图像}
  \end{subfigure}
  \begin{subfigure}{0.24\linewidth}
  \begin{minipage}[b]{1\linewidth}
  \includegraphics[width=1\linewidth]{figure/Pixel_cla_nyu/1gt.jpg}\vspace{4pt}
  \includegraphics[width=1\linewidth]{figure/Pixel_cla_nyu/2gt.jpg}\vspace{4pt}
  \includegraphics[width=1\linewidth]{figure/Pixel_cla_nyu/3gt.jpg}\vspace{4pt}
  \includegraphics[width=1\linewidth]{figure/Pixel_cla_nyu/4gt.jpg}\vspace{4pt}
  \includegraphics[width=1\linewidth]{figure/Pixel_cla_nyu/5gt.jpg}
  \end{minipage}
  \caption{真实深度图}
  \end{subfigure}
  \begin{subfigure}{0.24\linewidth}
  \begin{minipage}[b]{1\linewidth}
  \includegraphics[width=1\linewidth]{figure/Pixel_cla_nyu/1res.jpg}\vspace{4pt}
  \includegraphics[width=1\linewidth]{figure/Pixel_cla_nyu/2res.jpg}\vspace{4pt}
  \includegraphics[width=1\linewidth]{figure/Pixel_cla_nyu/3res.jpg}\vspace{4pt}
  \includegraphics[width=1\linewidth]{figure/Pixel_cla_nyu/4res.jpg}\vspace{4pt}
  \includegraphics[width=1\linewidth]{figure/Pixel_cla_nyu/5res.jpg}
  \end{minipage}
  \caption{本章方法}
  \end{subfigure}
  \begin{subfigure}{0.24\linewidth}
  \begin{minipage}[b]{1\linewidth}
  \includegraphics[width=1\linewidth]{figure/Pixel_cla_nyu/1eigen.jpg}\vspace{3.5pt}
  \includegraphics[width=1\linewidth]{figure/Pixel_cla_nyu/2eigen.jpg}\vspace{3.5pt}
  \includegraphics[width=1\linewidth]{figure/Pixel_cla_nyu/3eigen.jpg}\vspace{3.5pt}
  \includegraphics[width=1\linewidth]{figure/Pixel_cla_nyu/4eigen.jpg}\vspace{3.5pt}
  \includegraphics[width=1\linewidth]{figure/Pixel_cla_nyu/5eigen.jpg}
  \end{minipage}
  \caption{Eigen\cite{eigen2014depth}}
  \end{subfigure}
  \caption{像素分类单目深度估计网络在NYU depth数据集上的可视化结果,
  (a)为RGB图像,(b)为真实深度图,(c)为本章方法,(d)为
  Eigen\cite{eigen2014depth}结果。}
  \label{Pixel_cla_nyu_visualization_result}
  \end{figure*}
本小节展示了网络预测的可视化结果,KITTI与NYU depth的预测结果
分别在图\ref{Pixel_cla_KITTI_visualization_result}
与图\ref{Pixel_cla_nyu_visualization_result}中进行展示。
由于KITTI数据集提供的真实标注为稀疏的雷达点云,不便于
观察,实验中将
图\ref{Pixel_cla_KITTI_visualization_result}中的
真实标注进行了填充。本章方法与真实标注相比,具有相同
的深度层次结构以及基本的物体轮廓,网络在较远的情况
如第二行中的较远场景重建效果一般,未能将更远的
深度信息重建出来。与Liu\cite{liu2015learning}
相比,估计深度更为精确,错误像素较少,而且物体边缘更加
清晰。但是存在着边缘模糊,不锐利的问题。这可能是由于
方法在得到每个深度的置信度后使用
深度置信度与预测深度求期望操作造成的。

方法在NYU数据集上与Eigen\cite{eigen2014depth}
进行了对比,由于使用了更深的网络本章方法在预测结果上相比
\cite{eigen2014depth}精确很多,错误像素较少,预测效果
与真实标注十分相似。但是仔细观察会发现本网络
的结果仍存在着边缘预测错误的问题,如第三行
中的桌子边缘有着明显的错误,第四行图片中
的人体轮廓已经丢失,第五行的椅子形状
重建效果不好。而且整体仍然存在着模糊的问题。
\vspace{-0cm}

\subsection{特征注意力模块}
\begin{table*}[htb]
  \centering
  \caption{特征注意力模块指标在KITTI数据集上表现对比}
  \label{tab:FAM quantitative result}
  \begin{tabular}{c|cccc|ccc}
    \toprule
    \multirow{2}{*}{方法} & \multicolumn{4}{c}{数值越低效果越好}&\multicolumn{3}{|c}{数值越高效果越好}\\
    & Rel Abs & Rel Sq & RMSE& RMSE $log$ &$\delta<1.25$ &$\delta<1.25^2$ & $\delta<1.25^3$ \\   
    \midrule            
    FAM&0.117&0.142&4.643&0.198&0.809&0.952&0.964\\
    Upsample&0.123&0.152&4.772&0.207&0.796&0.951&0.952\\
    \bottomrule
  \end{tabular}
\end{table*}

本章设计了特征注意力模块,该模块应用在上采样过程中,
对不同来源的特征图进行加权并将它们应用在重建过程中。
本节通过实验对比来证明该模块对网络的性能提升。图\ref{FAM_visual}
为FAM模块视觉效果对比图。实验中采用了对特征图进行直接
双线性插值上采样作为对比。同样,本节对比了FAM与直接双线性插值上采样
的指标表现对比,如表\ref{tab:FAM quantitative result}所示,
添加FAM模块后各项指标均有一定的提升效果。
\begin{figure*}[htb]
  \centering
  \begin{subfigure}{0.24\linewidth}
    \begin{minipage}[b]{1\linewidth}
    \includegraphics[width=1\linewidth]{figure/Without_FAM/0.png}\vspace{4pt}
    \includegraphics[width=1\linewidth]{figure/Without_FAM/1.png}\vspace{4pt}
    \includegraphics[width=1\linewidth]{figure/Without_FAM/8.png}\vspace{4pt}
    \includegraphics[width=1\linewidth]{figure/Without_FAM/14.png}
    \end{minipage}
    \caption{RGB图像}
  \end{subfigure}
  \begin{subfigure}{0.24\linewidth}
    \begin{minipage}[b]{1\linewidth}
    \includegraphics[width=1\linewidth]{figure/FAM/0.png}\vspace{4pt}
    \includegraphics[width=1\linewidth]{figure/FAM/1.png}\vspace{4pt}
    \includegraphics[width=1\linewidth]{figure/FAM/8.png}\vspace{4pt}
    \includegraphics[width=1\linewidth]{figure/FAM/14.png}
    \end{minipage}
    \caption{真实标注}
  \end{subfigure}  
  \begin{subfigure}{0.24\linewidth}
  \begin{minipage}[b]{1\linewidth}
  \includegraphics[width=1\linewidth]{figure/Without_FAM/00000_colors.png}\vspace{4pt}
  \includegraphics[width=1\linewidth]{figure/Without_FAM/00001_colors.png}\vspace{4pt}
  \includegraphics[width=1\linewidth]{figure/Without_FAM/00008_colors.png}\vspace{4pt}
  \includegraphics[width=1\linewidth]{figure/Without_FAM/00014_colors.png}
  \end{minipage}
  \caption{无FAM}
  \end{subfigure}
  \begin{subfigure}{0.24\linewidth}
  \begin{minipage}[b]{1\linewidth}
  \includegraphics[width=1\linewidth]{figure/FAM/00000_colors.png}\vspace{4pt}
  \includegraphics[width=1\linewidth]{figure/FAM/00001_colors.png}\vspace{4pt}
  \includegraphics[width=1\linewidth]{figure/FAM/00008_colors.png}\vspace{4pt}
  \includegraphics[width=1\linewidth]{figure/FAM/00014_colors.png}
  \end{minipage}
  \caption{FAM}
  \end{subfigure}
  \caption{FAM模块对网络性能影响的视觉效果对比}
  \label{FAM_visual}
\end{figure*}
\section{本章小结}
深度估计任务可以建模为像素分类的问题,本章提出了一种
基于像素分类的单目深度估计网络,网络使用了编解码框架,
首先对真实的深度图进行
深度分类,使连续的深度图成为离散的深度区间标签,
进而可以引导网络预测每个像素落在不同深度区间的概率。
随后基于U-Net设计了一种编解码网络,与U-Net类似,
这种编解码网络中加入了短路连接,使编码端的特征图可以直接加入到
上采样重建中。为了使网络对重建过程中的编码端特征图和解码
特征图有所侧重的关注,本章提出了特征注意力模块,
实验结果表明像素分类算法
可以达到领先的重建精度,并且在细节处理上表现更为优异。
但是视觉结果表明该方法的预测图存在一定的模糊问题,这是可能是由于
最终的深度值是各个备选深度与其概率求得的期望深度。
另外像素分类的
类别数目作为一个超参数,仍需要大量的实验去确定最优值。
这些问题是将来需要继续进行研究解决的。
