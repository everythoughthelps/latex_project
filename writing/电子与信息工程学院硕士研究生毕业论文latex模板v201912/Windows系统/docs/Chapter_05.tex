\chapter{总结与展望}
深度估计为场景感知中的重难点问题,在众多的领域均有着广泛的应用。
本文为解决单目深度估计问题,基于深度学习方法提
出了两个行之有效的解决方案。

\section{工作总结}
第一章首先讨论了研究背景
和研究意义。当今针对深度估计有集中解决方案,分别是
硬件传感器,双目深度估计和单目深度估计。硬件传感器
法精度高,但是对硬件要求较高,仪器贵重,很难获得
密集的深度信息,而且能耗较高,搭载在无人智能体对能源同样是一项挑战。
双目深度估计有着严格的物理约束,鲁棒性较强,但是双目摄像头
很难做到完全同步,同时对基线等两个摄像头的相对关系要求极其严格。
两种方法的不足使学者们的研究目光转向了单目深度估计。单目深度估计
需要的设备只有一台消费级的相机,并且可以获得密集的深度图。但是
没有严谨的物理约束,所以对算法要求极高。随后探讨了国内外的研究现状,
对国内外的基于深度学习关键单目深度估计技术进行了梳理。分有监督学习,
半监督学习,无监督学习三个方向进行了客观的描述,介绍。

第二章主要是对基础知识进行描述,介绍。主要分为两个方面,一是
深度学习算法。首先介绍了深度学习的发展历史,经历了两次
低谷三次繁荣,依托算力,数据,网络结构三个因素最终在各个领域
均获得了极大地进步。其次介绍了深度学习算法的基础知识,由于
本文主要使用了卷积神经网络,所以围绕卷积神经网络算法展开描述。
包括算法流程:初始化权重,前向传播,反向传播,迭代优化,算法
中的基础概念等概述。而是深度估计问题。相比第一章重点描述
基于深度学习算法的单目深度估计技术,本章从更高的角度总结
深度估计任务。介绍了传感器深度估计,几何约束深度估计,
和单目深度估计技术的背景知识,测距原理,发展现状等。

第三章介绍了本文提出的第一种解决方案:基于像素分类的单目深度估计
网络。网络根据像素分类的原理,首先对真实的深度图进行
深度分类,使连续的深度图成为离散的深度区间标签,
进而可以引导网络完成预测每个像素落在不同深度区间的概率。
随后基于U-Net设计了一种编解码网络,与U-Net类似,
这种编解码网络中加入了短路连接,使编码端的特征图可以直接加入到
上采样重建中。为了使网络对重建过程中的编码端特征图和高维度
特征图有所侧重的关注,本章提出了特征注意力模块(Feature 
Attention Module,FAM),实验结果表明像素分类算法
可以达到领先的重建精度,并且在细节处理上表现更为优异。

第四章介绍了本文提出的第二种解决方案:自蒸馏单目深度估计框架。
为了提高网络的鲁棒性,提出使用室内外组合数据集来对网络进行训练。
但是随后的实验结果表明使用组合数据集会降低原本网络的性能。
这是因为复杂多样的数据分布对网络的拟合性能和表达能力是一项
巨大的挑战。
为了减弱这种退化现象,本章提出了自蒸馏单目深度估计框架。
该框架同时从室内数据集与室外数据集采样,然后将两批数据
分别送入学生编码器与负学生编码器。两个编码器共享权重,
他们分别提取两个数据集的特征,通过本章提出的
相异性损失的监督,来扩大两批特征的间距,即拉大两个数据集的类外间距。
随后特征图被送入解码器进行深度图重建,该过程在真实深度图
的引导下完成。实验表明,该框架可以很大程度上降低
编码器中卷积核的相似向量,且产生的特征图的相似性也大幅降低。
这说明框架对提升网络的表达能力有较大的帮助作用。详细的对比实验
表明,该框架可以减弱网络面对复合数据集时的衰退现象。
个别指标甚至更优于网络面对单一数据集时的表现。

\section{研究展望}
本文方法仍存在一定的不足,第三章提出的像素分类网络在室外数据集
上的表现仍有很大的提升空间,这是由于室外深度间距较大造成的。
在深度图重建中有一些细节预测结果并不好。同时像素分类的
类别数目作为一个超参数,仍需要大量的实验去确定最优值。
第四章相比其他领先方法在边界重建和深度精度上都表现良好。
但是与原方法相比,使用本框架时出现了一些锯齿化现象,
这可能是由于添加了自适应池化层造成的,这需要
去寻找更优秀的解决方案。通过放大误差图可以发现自蒸馏框架的加入
使得网络在远距离的预测误差大幅减少。该框架适用于所有的
编解码网络,移植性强,但是方法只适用于两个组合数据集,当面对
更多类别的数据集时无法操作,未来将会针对这些不足进行
研究。