\documentclass{assignment}
\usepackage{amsmath}
\usepackage[ruled,vlined]{algorithm2e}
\usepackage{listings}
\usepackage{color}

\definecolor{dkgreen}{rgb}{0,0.6,0}
\definecolor{gray}{rgb}{0.5,0.5,0.5}
\definecolor{mauve}{rgb}{0.58,0,0.82}

\lstset{frame=tb,
  language=Python,
  aboveskip=3mm,
  belowskip=3mm,
  showstringspaces=false,
  columns=flexible,
  basicstyle={\small\ttfamily},
  numbers=none,
  numberstyle=\tiny\color{gray},
  keywordstyle=\color{blue},
  commentstyle=\color{dkgreen},
  stringstyle=\color{mauve},
  breaklines=true,
  breakatwhitespace=true,
  tabsize=3
}

%\coursetitle{Creating assignments}
%more class  https://ctan.org/topic/class
%more template https://www.latexstudio.net/
\courselabel{Stochastic Process}
\exercisesheet{Home Work 9}{Documentation}
\student{Pan Meng}
\semester{Fall 2021}
\date{Dec 30, 2021}
\finishdate{Dec 28, 2021}
%\usepackage[pdftex]{graphicx}
%\usepackage{subfigure}
\SetKw{KwBy}{by}
\begin{document}
% 根据输入的第一个参数围来读取图片显示的写法
\logo{logo.jpg}
% 固定死是logo.jpg的写法
% \logo

\begin{problemlist}
\pbitem 
A Manufacturer at each time period receives an order for 
her product with probability $p$ and receives no order with 
probability $1-p$.
At any period, she has a choice of processing all the unfilled orders 
in a batch, or process no order at all. The maximum number of orders that 
can remain unfilled is $n$.
The cost per unfilled order at each time period is $c > 0$, 
the setup cost to process the unfilled orders is $K > 0$.
The manufacturer wants to find a processing policy that minimizes 
the total expected cost with discount factor $\alpha<1$

\begin{problem}
\begin{answer}
  \vspace{-1cm}
\begin{flushleft}
  \large\textbf{1. model formulation}
\end{flushleft}


\begin{flushleft}
  \large\textbf{2. pseudocode}
\end{flushleft}

\begin{flushleft}
  \large\textbf{3. code}
\end{flushleft}
\begin{lstlisting}
  class DiscountProblem():

    def __init__(self, c, K, n, p, a):
        self.c = c
        self.K = K
        self.n = n
        self.p = p
        self.a = a
        self.action_prob = {0: 0.5, 1: 0.5}
        self.transition = self.__init_transition()
        self.V = [0 for _ in range(n + 1)]

\end{lstlisting}

\end{answer}
\end{problem}

\pbitem What does the preamble contain ?

The preamble may contain the following declarations\footnote{Current markup's preamble.} :

\begin{verbatim}
\documentclass{assignment}
\coursetitle{Creating assignments}
\courselabel{ASG 101}
\exercisesheet{Home Work 1}{Documentation}
\student{Madhusudan Singh}
\semester{Summer 2004}
\date{July 14, 2004}
%\usepackage[pdftex]{graphicx}
%\usepackage{subfigure}
\end{verbatim}


Its possible. Just pass the options in the preamble :

\begin{verbatim}
\documentclass[option1,option2, ...]{assignment}
\end{verbatim}


Equation numbers refer to the problem number. For instance,

\begin{verbatim}
\begin{eqnarray}
E & = & mc^{2} \label{eqn:emc2} \\
\textrm{That is how equations are numbered} \ldots \label{eqn:numbered} \\
\textrm{Or not numbered} \ldots \nonumber
\end{eqnarray}
\end{verbatim}

\begin{eqnarray}
E & = & mc^{2} \label{eqn:emc2} \\
\textrm{That is how equations are numbered} \ldots \label{eqn:numbered} \\
\textrm{Or not numbered} \ldots \nonumber
\end{eqnarray}


\begin{verbatim}
\begin{answer}
\begin{eqnarray}
Answer=f(bold) \nonumber
\end{eqnarray}
\end{answer}
\end{problem}
\end{verbatim}


\pbitem Can one have more than one answer section for the problem ?

\begin{problem}

Most definitely.

Certain problems have many parts :

\begin{enumerate}
\item Part 1

\begin{answer}
Answer to part one.
\end{answer}

\item Part 2

\begin{answer}
Answer to part two.
\end{answer}

\end{enumerate}

\begin{verbatim}
\begin{enumerate}
\item Part 1

\begin{answer}
Answer to part one.
\end{answer}

\item Part 2

\begin{answer}
Answer to part two.
\end{answer}

\end{enumerate}

\end{verbatim}

\end{problem}


\pbitem What are the copyright conditions for this class file ?

\begin{problem}

This material is subject to the \LaTeX\ Project Public License. See http://www.ctan.org/tex-archive/help/Catalogue/licenses.lppl.html for the details of that license. See the LICENSE file for more details.


\begin{answer}
\begin{eqnarray}
&& Q. E. D. \nonumber
\end{eqnarray}
\end{answer}

\end{problem}


\pbitem How do I get help using this class ?
\begin{problem}


You may post your queries on comp.text.tex . I check it fairly regularly.
\end{problem}

\end{problemlist}

\end{document}
